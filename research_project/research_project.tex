% interactcadsample.tex
% v1.03 - April 2017

\documentclass[]{interact}

\usepackage{epstopdf}% To incorporate .eps illustrations using PDFLaTeX, etc.
\usepackage{subfigure}% Support for small, `sub' figures and tables
%\usepackage[nolists,tablesfirst]{endfloat}% To `separate' figures and tables from text if required

\usepackage{natbib}% Citation support using natbib.sty
\bibpunct[, ]{(}{)}{;}{a}{}{,}% Citation support using natbib.sty
\renewcommand\bibfont{\fontsize{10}{12}\selectfont}% Bibliography support using natbib.sty

\theoremstyle{plain}% Theorem-like structures provided by amsthm.sty
\newtheorem{theorem}{Theorem}[section]
\newtheorem{lemma}[theorem]{Lemma}
\newtheorem{corollary}[theorem]{Corollary}
\newtheorem{proposition}[theorem]{Proposition}

\theoremstyle{definition}
\newtheorem{definition}[theorem]{Definition}
\newtheorem{example}[theorem]{Example}

\theoremstyle{remark}
\newtheorem{remark}{Remark}
\newtheorem{notation}{Notation}


% tightlist command for lists without linebreak
\providecommand{\tightlist}{%
  \setlength{\itemsep}{0pt}\setlength{\parskip}{0pt}}



\usepackage{hyperref}
\usepackage[utf8]{inputenc}
\def\tightlist{}


\begin{document}


\articletype{ARTICLE TEMPLATE}

\title{Potências Médias - projeto de pesquisa}


\author{\name{M. Galdino$^{a}$}
\affil{$^{a}$Political Science Department, Universidade de são Paulo,
Brazil.}
}

\thanks{CONTACT M.
Galdino. Email: \href{mailto:mgaldino@usp.br}{\nolinkurl{mgaldino@usp.br}}}

\maketitle

\begin{abstract}
O conceito de potências médias, apesar de sua prevalência na literatura
de Relações Internacionais (RI), enfrenta um desafio crítico de
ambiguidade conceitual que limita sua utilidade analítica. Esta pesquisa
visa abordar esse desafio propondo um novo arcabouço teórico que integra
a teoria dos jogos não cooperativa, especificamente através da adaptação
dos conceitos de estratégias complementares e substitutas. Ao fazer
isso, o projeto tem como objetivo superar as divisões teóricas
existentes e promover uma compreensão mais precisa e abrangente das
potências médias. Tradicionalmente, a RI concentrou-se na teoria do
comportamento das grandes potências, negligenciando uma teoria inclusiva
que explique o comportamento de estados de diferentes magnitudes. Este
estudo argumenta pela necessidade de uma teoria do comportamento estatal
que englobe não apenas as grandes potências, mas também as médias e
pequenas, oferecendo uma maneira empírica de diferenciar esses grupos de
países. Ao unificar as variáveis explicativas relacionadas ao poder,
comportamento e identidade dos estados, propomos um caminho para uma
validação empírica robusta, estabelecendo um programa de pesquisa
progressivo no sentido Lakatos. Além de sua contribuição teórica, este
projeto delineia um desenho de pesquisa empírica, priorizando o estudo
de cabos diplomáticos como fontes primárias. Essas fontes são
consideradas essenciais para testar as hipóteses derivadas de nossa
abordagem teórica, possibilitando uma avaliação detalhada das
estratégias adotadas pelas potências médias no cenário internacional.
Assim, o projeto não apenas promete clarificar a categorização de
potências médias, mas também fornece uma base para futuras investigações
empíricas neste campo, contribuindo significativamente para a literatura
existente e oferecendo novas ferramentas analíticas para pesquisadores e
formuladores de políticas.
\end{abstract}

\begin{keywords}
Potências Médias; Teoria dos Jogos; Estratégias Complementares;
Estratégias Substitutas; Cabos Diplomáticos;
\end{keywords}

\hypertarget{introduuxe7uxe3o}{%
\section{Introdução}\label{introduuxe7uxe3o}}

A Teoria das Potências Médias, desde seu início (cf.
\citet{chaudhuri_1969}; \citet{holbraad_71}), enfrentou problemas de
clareza analítica. A dificuldade em conceituar quem seriam as potências
médias, seja por alguma medida objetiva de poder (combinando um ou mais
indicadores) ou por seu comportamento, é reveladora dessa confusão
analítica. A despeito disso, a literatura continua a usar o conceito,
variando na forma de defini-lo e operacionalizá-lo empiricamente, assim
como nas várias tentativas de resolver os problemas conceituais.

A concepção mais tradicional de potência média é baseada na posição que
os estados ocupam no sistema internacional (por exemplo,
\citet{holbraad_84}; \citet{cooper_etal_93}; \citet{shin_12}). Assim,
países com capacidades médias ou medianas teriam um comportamento de
política externa diferente ou previsível em relação a estados com maior
ou menor capacidade. Essa diferenciação pode se dar tanto pela formação
de interesses distintos quanto pelo estilo ou estratégia de atuação na
esfera internacional (cf. \citet{cooper_11} para uma revisão da
literatura). No entanto, isso não é suficiente para operacionalizar o
conceito de maneira consistente, uma vez que não há clareza sobre qual
medida de capacidade deve ser usada (seja ela militar, econômica,
diplomática ou uma combinação dessas e de outras variáveis), a posição
relativa pode mudar dependendo da região geográfica considerada e a
inclusão de variáveis como participação em alianças, ao modificar a
medida de poder, acaba trazendo confusão analítica ao conceito
(\citet{cooper_11}). O resultado dessa confusão é a inexistência de uma
lista definida, no tempo e espaço, de quais países seriam potências
médias, o que torna inviável testar se, de fato, há um comportamento
distintivo na política externa desses países.

Em segundo lugar, há teorias focadas no comportamento desses estados. O
que seria distintivo das potências médias seria seu comportamento, tal
como a adesão ao multilateralismo, a tentativa de agir como ``bom
cidadão'', a capacidade de atuar como mediador de conflitos, etc.
(\citet{schiavon_dominguez_16}; \citet{stephen_13}; \citet{welsh_04}).

O problema óbvio dessa abordagem é a impossibilidade de explicar
comportamento com base em uma categorização do comportamento devido à
óbvia circularidade. Além disso, essa abordagem tampouco contribuiu, ao
menos, para observar um padrão de comportamento dos países
tradicionalmente concebidos como potências médias. Alguns estudos
combinam capacidades e comportamento para classificar os países, mas
isso não resolve o problema de circularidade (\citet{cooper_11}).

Por fim, mais recentemente, tem havido tentativas de conceituar potência
média a partir da identidade dos estados, em particular aqueles que se
autodefinem como potências médias (\citet{hynek_07};
\citet{gecelovsky_09}; \citet{debhal_23}).

Esse espectro conceitual sugere uma continuidade com a tradição da RI no
estudo das grandes potências. Tradicionalmente, a RI focou em uma teoria
do comportamento das grandes potências, deixando um vazio analítico
sobre o comportamento estatal além dessas. A categorização qualitativa
de ``grande potência'' ou ``superpotência'' indica que outras
categorias, como as ``potências médias'', podem ser analiticamente
úteis, embora a sua utilidade ainda seja um ponto de debate intenso.

Ante a complexidade desse cenário, parece claro que, para redescobrir a
utilidade analítica do conceito de potências médias, é necessário um
trabalho teórico e empírico de grandes proporções. Do lado teórico,
parece-nos claro que qualquer que seja o marcador ou marcadores que
distingam potências médias, essas variáveis precisam estar relacionadas
também ao comportamento dos estados de outros tamanhos. Isso sugere que
a pesquisa empírica nessa área precisa ser comparativa e incluir não
apenas as pretensas potências médias, mas também os países que estariam
em outras categorias.

O presente trabalho, portanto, parte dessa ambição como horizonte de
pesquisa. Por outro lado, este é um projeto laborioso e que exigirá
esforço colaborativo para ser empreendido. Assim, o que apresentamos a
seguir é nossa estratégia para um primeiro passo nessa direção, sem a
pretensão de delinear o caminho todo a ser percorrido. Além de ser mais
realista, parece-nos mais efetivo, na medida em que facilita a correção
de rotas que o trabalho de pesquisa e as descobertas não antecipadas
demandam.

No nível teórico, o presente trabalho propõe uma solução ancorada em uma
adaptação dos conceitos de estratégias complementares e substitutas da
teoria dos jogos não cooperativa. Esta abordagem promete superar a
ambiguidade conceitual em torno das potências médias, unificando os
comportamentos dos estados numa única teoria que explique a
diferenciação entre os três grupos de países - grandes, médias e
pequenas potências. Argumentamos que tal teoria, ao enfatizar a
centralidade das relações entre estratégias dos países em diferentes
arenas de conflito (se complementares ou substitutas), abre caminho para
uma nova agenda de pesquisa empírica.

A perspectiva que adotamos aqui não é inteiramente nova, já que a
interconexão entre jogos distintos e suas consequências nas decisões dos
jogadores é um conceito que já foi explorado no campo das Relações
Internacionais, particularmente na teoria de jogos de dois níveis de
Putnam (1988) e suas subsequentes ampliações e aplicações. Entretanto,
nosso framework se distingue ao aplicar-se a qualquer conexão entre
jogos distintos, não apenas entre o nível doméstico e externo,
proporcionando insights únicos sobre a dinâmica de estratégias
complementares e substitutas.

Outra abordagem relacionada é a de ``issue linkages'' (Tollison \&
Willett, 1979), que embora compartilhe algumas semelhanças com nosso
framework, difere na medida em que foca na construção de acordos
multidimensionais que geram benefícios mútuos, sem necessariamente
antecipar as complementaridades ou substitutibilidades estratégicas.
Assim, embora haja precedentes, nossa abordagem teórica é pioneira no
campo das relações internacionais.

Do ponto de vista empírico, pretendemos construir uma base de dados de
cabos diplomáticos de países em perspectiva comparada, consistindo de
países dos mais variados tamanhos. A recente proliferação das leis de
acesso à informação e iniciativas de dados abertos mundo afora
(\citet{zuffova_20}) possibilita que, finalmente, tal tarefa possa ser
empreendida com boa chance de sucesso.

Este estudo busca, portanto, consolidar a teoria de potências médias
como um programa de pesquisa progressivo, no sentido proposto por
Lakatos, mediante a conexão de variáveis explicativas - sejam elas
relativas a poder, comportamento ou identidade - com hipóteses causais
testáveis empiricamente.

A abordagem proposta oferece um caminho promissor para a superação dos
desafios conceituais e metodológicos que têm caracterizado o estudo das
potências médias até o momento. Por fim, pretendemos ilustrar essa
perspectiva com um estudo de caso focado no Brasil e EUA, estudando os
cabos diplomáticos disponíveis para ambos os países.

Nas próximas seções, desenvolvemos um pouco mais o arcabouço teórico,
para ilustrar o tipo de teorização que será realizada, e, em seguida,
apresentamos como se dará a pesquisa empírica no estudo de caso que
pretendemos realizar.

\hypertarget{modelo-teuxf3rico}{%
\section{Modelo Teórico}\label{modelo-teuxf3rico}}

Em um artigo seminal sobre a aplicação da teoria dos jogos à economia,
\citet{bulow_etal_85} introduziu a ideia de jogos com estratégias
complementares ou substitutas. Considerando as estratégias como
variáveis contínuas (e.g., nível de investimento), estratégias são ditas
complementares se o aumento no nível de uma estratégia por um agente
torna ótimo para outro agente aumentar o nível de sua própria
estratégia. Matematicamente, isso ocorre quando a derivada parcial
cruzada do payoff de um agente, em relação à sua estratégia e à do outro
jogador, é positiva. Em contraste, estratégias são substitutas quando o
aumento na estratégia de um agente leva o outro a reduzir a sua,
evidenciado por uma derivada parcial cruzada negativa.

O exemplo arquetípico apresentado por \citet{bulow_etal_1985} envolve
dois mercados, \(1\) e \(2\), com demandas independentes onde a empresa
\(A\) monopoliza o primeiro mercado e compete com a empresa \(B\) no
segundo. Como os produtos de \(A\) e \(B\) no mercado \(2\) são
substitutos, o modelo de equilíbrio de Cournot demonstra que um
comportamento mais agressivo de \(A\) no mercado \(1\) induz a empresa
\(B\) a ser menos agressiva no mercado \(2\), evidenciando a
substitutividade das estratégias.

Esse modelo tem implicações significativas ao ser transportado para o
contexto das Relações Internacionais, sugerindo que o ganho de um estado
\(A\) derivado de uma mudança em um jogo \(1\) é afetado pela sua
posição em um jogo \(B\), seja como grande potência (monopolista),
potência média (oligopolista) ou pequena potência (competidor puro).

A aplicação deste arcabouço às Relações Internacionais permite discutir
como grandes potências distinguem-se das potências médias pela
capacidade de alterar os termos do jogo \(1\), induzindo comportamentos
específicos nos países envolvidos. Da mesma forma, potências médias
podem adotar comportamentos de coalizões empreendedoras de norma
(\citet{ravenhill_2018}), alterando exogenamente os custos para países
em um jogo A, para induzir mudanças de comportamento em um jogo B.

Para ilustrar o potencial analítico desse arcabouço, propomos um esboço
informal de dois jogos distintos, demonstrando como a presença de
estratégias complementares ou substitutas explica certos comportamentos
e auxilia na categorização dos tamanhos das potências.

Diferentemente da literatura existente, nossa abordagem torna a
categorização de potência média dependente da estrutura da conectividade
dos jogos. Assim, podemos clarificar a confusão conceitual presente na
definição de potências médias na literatura, utilizando medidas de
capacidade que influenciam a estrutura (complementar ou substituta) das
estratégias, enquanto incorporamos variáveis contextuais (como a
participação em alianças) e a identidade dos atores, na medida em que
afetam suas preferências. Além disso, ao propor um mecanismo específico,
podemos testar empiricamente as hipóteses derivadas dos modelos sem
incorrer em circularidade argumentativa.

\hypertarget{modelos-de-estratuxe9gias-complementares-e-substitutas}{%
\subsection{Modelos de Estratégias Complementares e
Substitutas}\label{modelos-de-estratuxe9gias-complementares-e-substitutas}}

Considere um jogo entre dois estados, \(A\) e \(B\), disputando um
território de valor \(X = [0,1]\), conforme proposto por Fearon (1995).
O estado \(A\) prefere um resultado próximo de \(1\), enquanto \(B\)
favorece um resultado próximo de \(0\). Na negociação diplomática para o
conflito, o resultado será denotado por \(x \in X\), com as utilidades
\(u_A(x)\) para \(A\) e \(u_B(1-x)\) para \(B\). Cada estado possui uma
utilidade esperada da guerra dada por: \(p_a u(1) + (1-p)u(0) - c_a\)
para \(A\) e uma fórmula similar para \(B\), com custo \(c_b\). Segundo
o modelo de barganha de Rubinstein, com horizonte infinito onde as
partes fazem ofertas alternadas e excluindo temporariamente a opção de
guerra, o equilíbrio de Nash perfeito em sub-jogo (sem a opção de
guerra) seria dado por \((x, 1-x)\), onde
\(x = \frac{1 -\delta_A}{1 - \delta_A \delta_B}\). A opção da guerra, no
entanto, restringe os valores de \((x, 1-x)\) que são preferíveis à
guerra, com \(x > \mathbb{E}[u^g_A]\) e \(1 - x > \mathbb{E}[u^g_B]\)
para que a opção diplomática seja estritamente preferível para ambos.

A utilidade de estratégias complementares ou substitutas neste contexto
torna-se evidente ao considerar diferentes cenários envolvendo grandes
potências, potências médias e países pequenos. Por exemplo, em um
cenário onde o estado \(A\) é uma grande potência, a decisão sobre a
alocação de recursos militares próximo à fronteira em conflito pode
aumentar a sua probabilidade de vitória, influenciando assim as
estratégias de negociação.

Em contraste, em um conflito sobre a expropriação de uma empresa
estrangeira, como um setor de extração de petróleo, a presença de um
aliado do país \(B\) pode compelir o país \(A\) a ceder mais na
negociação para não prejudicar a aliança com o país \(C\). Isso ilustra
como estratégias em um jogo podem influenciar os resultados de outro.

De maneira geral, o objetivo destes dois esboços de jogos é ilustrar que
o comportamento dos estados é influenciado pela interconexão entre
diferentes jogos e pela situação específica em que se encontram os
países. Isto é, as estratégias ótimas de um país podem ser manipuladas
por terceiros (superpotências) ou, alternativamente, o próprio país pode
ter a oportunidade de manipular as estratégias de outros países (como
potências médias em contextos regionais). Portanto, o que distingue uma
superpotência de uma potência média não é apenas a capacidade de
influenciar as estratégias ótimas de terceiros, mas sim o contexto em
que essa influência ocorre: enquanto uma superpotência pode exercer sua
influência em conflitos de qualquer escala, uma potência média pode
fazê-lo predominantemente em contextos de conflitos regionais.

Por outro lado, uma superpotência pode se encontrar em uma situação
similar à de um monopolista, conforme descrito por
\citet{bulow_etal_85}, adotando uma estratégia menos agressiva devido à
conexão entre diferentes arenas (jogos), precisamente porque detém a
posição de superpotência. Alternativamente, pode adotar estratégias de
construção multilateral, que são típicas de potências médias, se essas
representarem o equilíbrio do jogo. Deste modo, sugerimos que não existe
um comportamento intrínseco ou típico às potências médias, grandes ou
pequenas. O que determina o comportamento de cada tipo de país é a
frequência e a natureza dos jogos enfrentados, que incentivam certos
tipos de respostas como estratégias ótimas. Assim, esse arcabouço
teórico ajuda a explicar a dificuldade em identificar um padrão
simplificado de comportamento entre as diferentes potências.

\hypertarget{pesquisa-empuxedrica}{%
\section{Pesquisa Empírica}\label{pesquisa-empuxedrica}}

Conforme propomos, o avanço na teoria das potências médias exige o
desenvolvimento dentro de uma teoria mais abrangente sobre o
comportamento estatal, que estabeleça ligações claras entre diferentes
grupos de países e seus comportamentos previsíveis na arena
internacional. Para validar a utilidade analítica da categoria de
potências médias, é necessário caracterizar as estratégias dos agentes
de forma precisa. Uma pesquisa empírica, orientada teoricamente, é
essencial para testar hipóteses relacionadas às características dos
jogos em que os países estão inseridos e seus comportamentos. Esta
abordagem permite, posteriormente, agrupar essas características em
clusters identificáveis, facilitando a distinção de potências médias.

Como ponto de partida para nossa investigação, escolhemos analisar as
relações diplomáticas entre Brasil e Estados Unidos em relação à China.
A seleção destes dois países fundamenta-se tanto em razões pragmáticas
--- a disponibilidade de cabos diplomáticos --- quanto teóricas, ao
considerar que representam, respectivamente, uma superpotência e um
potencial candidato a potência média. Se a teoria que defendemos se
mostrar aplicável, esses casos específicos proporcionarão um terreno
fértil para sua validação. Embora isso não garanta a generalização da
teoria para outros casos, iniciar a pesquisa por esses candidatos
promissores é uma estratégia metodologicamente mais frutífera do que a
escolha de casos mais complexos e que exijam mais nuances e
qualificações.

A escolha por analisar cabos diplomáticos decorre de sua capacidade de
revelar as preocupações mais autênticas e sinceras dos países
envolvidos, diferenciando-se dos comportamentos que são observáveis
publicamente e, por vezes, estrategicamente calculados. Esses documentos
oferecem uma oportunidade ímpar para testar hipóteses derivadas da
modelagem de jogos complementares e substitutos.

A título de exemplo, especificamos os tipos de hipóteses que temos em
mente. Podemos derivar, a partir do desenho de jogos com estratégias
substitutas, que países capazes de manipular condições exógenas do jogo
o farão para induzir o oponente a adotar estratégias menos agressivas.
Portanto, é razoável esperar que tais movimentações estratégicas ocorram
entre potências médias com uma frequência distinta em comparação às
grandes potências e superpotências. Isso nos permite formular a seguinte
hipótese empírica para ser testada:

\(H_a\): A frequência com que superpotências manipulam as estratégias de
outros países é tanto maior quanto maior o poder do país.

A análise de cabos diplomáticos, cruzada com medidas tradicionais de
poder, permitirá testar essa hipótese empiricamente. Será essencial
desenvolver uma metodologia para mensurar, nos cabos diplomáticos, a
manipulação de estratégias. Se operacionalizada com êxito, essa variável
nos permitirá testar a correlação entre poder e frequência de
manipulação, assim como facilitar a categorização dos estados em
clusters conforme a frequência de manipulação. Esperamos, assim,
distinguir potências médias de superpotências e pequenos estados.

No contexto atual do estudo, a predominância desse comportamento por
parte dos EUA em relação ao Brasil pode oferecer poucas conclusões
novas, dada a previsibilidade do resultado. O objetivo desta
investigação, no momento, é demonstrar como realizar a análise de forma
consistente e rigorosa, preparando o terreno para expandi-la a mais
países e, assim, explorar plenamente o potencial de nossa abordagem. A
clareza dos resultados esperados serve como um teste de validação para o
tratamento empírico dos dados.

Uma segunda hipótese, derivada de nosso framework, considera a
importância geográfica dos conflitos. Podemos classificá-los, de forma
binária, como regionais ou globais/multiregionais. Presumimos que
superpotências possuam capacidade de manipulação nos dois contextos,
enquanto as potências médias tenham essa capacidade limitada a questões
regionais. Assim, formulamos:

\(H_{b1}\): A frequência relativa com que potências médias manipulam
jogos é maior em questões regionais do que em globais, diferença não
observada em superpotências.

\(H_{b2}\): A frequência relativa com que potências médias manipulam
jogos é maior em questões regionais do que em globais, enquanto para
superpotências, a relação é inversa.

Observa-se que grandes potências, em alguns casos contraintuitivamente
classificadas como potências médias por alguns autores (por exemplo,
Alemanha, Reino Unido)\footnote{cf. \citet{trommer_17}, \citet{otte_99}},
podem não se distinguir significativamente das chamadas potências
médias. Embora esta investigação não explore empiricamente tal distinção
no momento, a inclusão de mais casos permitirá uma análise mais
detalhada dessa hipótese discutida na literatura.

\bibliographystyle{tfcad}
\bibliography{Middle-Powers-bib.bib}


\input{"appendix.tex"}



\end{document}
